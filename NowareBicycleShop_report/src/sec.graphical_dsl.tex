\section{Graphical DSL using GMF}
\label{sec.gmf}
\noindent \emph{GMF} is a framework for building modeling-like graphical
Eclipse-based editors, supporting the generation of a graphical editor from a
set of models. Using the GMF dashboard, we
derived the \emph{Tooling Definition Model} and \emph{Graphical Definition
Model}. By combining these two we obtained the \emph{Mapping Model} which helped
us generate the \emph{Diagram Generation Model}. Firstly, we replaced the
default icons, in the \emph{edit} plug-in with icons more relevant to our bike model.
Than, in the \emph{.gmfgraph}, we modified the properties, such as size and
background, for each figure. In the \emph{.gmfmap} file, we added feature label
mappings for all the fields we want to visualize while adding parts in the
graphical editor. Finally, we have launched the runtime environment and created
three custom bicycle models with the graphical editor. The examples are depicted
in appendix \ref{appendix.gmf_models}.\\

\noindent We implemented the validation in GMF as \emph{audit rules} using the
OCL constraints we have created in the Ecore model. If a concrete bicycle model
violates these constraints, messages will be displayed in the problem view, when
validating the model, as depicted in appendix \ref{appendix.gmf_models}.\\

\noindent Feature initializers were not a part of the implementation, however
they could be useful. Every new bicycle model could have additional parameters
with name and value set added on the initialization. For instance, this could be
a new parameter named 'Telephone Number' having an empty default value. Modeling
link constraints were neither part of implementation, since we do not use links
in the project. Nonetheless, it is a useful feature that allows to constrain
users from linking objects together.\\