\section{Graphical DSL using GMF}
\label{sec.gmf}

\noindent
The Graphical Modeling Framework makes it possible to provide a 
graphical visualization of the model through a graphical editor. 
The user is shown a canvas where elements of the model can be dragged 
and dropped from the palette. This is a more intuitive way to interact 
with the system for customers of NBS.\\
At this point there are two ways to create a new bicycle model. 
Either by creating a tree based editor where the user has to add 
parts of the bike as children to the Bicycle node. The second option 
is by using the GMF Editor where the user selects the part (which also 
has an appropriate icon) on the palette and drops it on the canvas.
There are no connections between the certain parts of the bicycle, 
because they can be placed in any order and different bicycles might 
have different amount of parts.\\
The user has the possibility to change the name of a part but changing 
the price for obvious reasons is only possible in the Properties view. 
For that reason in the final implementation the Properties view should 
not be shown to the customer. The OCL Constraints as mentioned earlier 
are implemented in GMF, specifically in the gmfmap as Audit Rules.\\
//TODO: Feature initializers
//TODO: Modeling constraints
