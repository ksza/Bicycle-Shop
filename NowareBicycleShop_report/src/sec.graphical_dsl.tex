\section{Graphical DSL using GMF}
\label{sec.gmf}

\noindent \emph{GMF} is a framework for
building modeling-like graphical Eclipse-based editors.
\cite{Article_Introducing_GMF} GMF supports the generation of a graphical editor
from a set of models. In this section we will present the important
milestones of creating a graphical editor and why were they necessary. 

\noindent Earlier we described how we created the \emph{Ecore} domain model. The
\emph{Ecore} model defines the non-graphical information managed by the editor. The first
model needed for the graphical editor is the \emph{Graphical Definition Model}
(gmfgraph) which is used to define graphical elements (figures, nodes, links)
which will be displayed in the editor. \\

\noindent The \emph{Tooling Definition Model} (gmftool)
is used to specify the palette, creation tools and actions for the
graphical elements. The \emph{Mapping Model} (gmfmap) is combined of the
three previously created definition models (domain, graphical and tooling).
It provides the input for the transformation step. The OCL Constraints are
implemented in the \emph{Mapping Model} as Audit Rules. The \emph{Mapping Model}
is then transformed into the \emph{Diagram Generation Model} (gmfgen). This
model is used for generating the code for the graphical editor. It is similar to
the \emph{EMF Generation Model}, mentioned earlier. \cite{GMF_Tutorial}\\

\noindent Having all the required models it is possible to generate the code
necessary for a running diagram editor. When the user creates a new NBS diagram he is
shown a canvas where elements of the model can be dragged and dropped from the
palette. There are no connections between the certain parts
of the bicycle, because they can be placed in any order and different bicycles
might have different amount of parts. This is an intuitive way to interact
with the system for the customers of NBS.\\

\noindent The user has the possibility to change the name of a part but changing
the price is not allowed in the canvas. It is only possible to change the
price in the Properties view. For that reason the Properties view should not be
shown to the customer in the final implementation.  
Validation error examples are shown in Figure \ref{fig.gmf_validation}.\\

\noindent Feature initializers were not a part of the implementation, however
they could be useful. Every new bicycle model could have additional parameters
with name and value set added on the initialization. For instance, this could be
a new parameter named 'Telephone Number' having an empty default value.\\
Modeling link constraints were neither part of implementation, since we do
not use links in the project. Nonetheless, it is a useful feature that allows
to constrain users from linking objects together. \\