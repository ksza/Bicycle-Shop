\section{The Model}
\label{sec.logical_model}

\subsection{Conceptual Model}
\noindent When creating the models firstly the highest level of 
abstraction was used. The relationships between the important concepts 
of the model were roughly described in a conceptual model (Figure 1). 
The different parts and other characteristics were identified which 
constitute a bicycle. There is a hierarchy between these parts which 
also represents the associations between those objects. The conceptual 
model can also be expressed by statements, e.g.: "A bicycle consists of 
several bicycle parts. The parts can have common attributes which are 
inherited from the super part type or have their own specific attributes 
with the appropriate data type. 

\subsection{Logical Model}
\noindent The conceptual model was refined in the logical model (Figure 1) where
all the important classes, relations and attributes were identified. 
All the design choices were included by taking into consideration NBS' 
description of a custom bicycle.\\

\noindent The central class of the model is the Part class. It is an abstract
class and all the concrete parts of the bicycle extend this. The abstract Node 
class was introduced to conform to the convention of having a Node Super 
Type presented by the exercises solutions during the course. 
The Part has all the default attributes (price, manufacturer and id) 
that all the parts of the bicycle are supposed to have. Enumeration types 
were used to represent the various types of parts NBS provides to their 
customers. These are defined in the FrameType, SaddleType, Color, 
PedalTypes, Manufacturers classes.\\

\noindent A note on the Color class; although the description states that the 
frame can be painted in any color enumerating the different colors was 
found to be the most suitable solution. In order to provide more 
flexibility to the users of the system a Color could have been 
represented by the three RGB values in a 0-255 range, but it was 
after discussing this option the group found it too circumstantial 
both to develop and the users to use as it would require an additional 
tool to provide an RGB value to a certain color.\\

\noindent A note on the InnerGear and OuterGear classes; as opposed 
to using enumerations in previous cases, this time subtypes were used 
for the reason that there are only two types of gears on bikes and 
it is not expected to change. This implies that the Gear class is an 
abstract class.\\

\noindent A note on the Bicyle, CustomerData and Parameter classes; 
the Bicycle class represents the customized bike itself which is 
built up from the different parts. It has a Price attribute which 
will be calculated by summing up the prices of individual parts. 
CustomerData captures the basic information about the customer 
ordering the bike and it was decided to be extracted from the 
attributes of Bicycle to have a better modularity and overview of 
the model. It is possible to extend the CustomerData and Bicycle 
classes via the Parameter class which has name-value pairs in case 
it is needed.\\

\noindent References from Parts, CustomerData and Parameter to Bicycle 
are of a composite relationship because all these information have to 
be connected to an instance of a Bicycle.\\

\subsection{Modeling Layers}

\noindent Finally a Metamodel Layer Schema (Figure 3) shows the connections
between the elements of the different meta levels. On the top M3 layer is
the Ecore meta-meta model to which the M2 level meta model conforms. 
M2 is the logical model developed for the NBS case and all its elements are
derived from the meta-meta model. E.g. the Frame is a type of EClass. There is 
EReference between the Bicycle and the Part stating that the Bicycle can have
more parts. The name in the CustomerData class is of EAttribute having a 
String as an EDataType.\\
The M1 level model represents the model of a concrete bicycle. This is 
the level the employees of NBS and the customers work. This model also
conforms to the higher M2 level and all the terms used here can be found 
in the logical model.\\
M0 level represents the physical bike which will be delivered to the
customer who ordered it.\\
There are three custom bicycle models created based on the Ecore model. These
are Figure 4, Figure 5 and Figure 6 in the Appendix.
