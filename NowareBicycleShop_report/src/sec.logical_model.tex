\section{The EMF Model}
\label{sec.logical_model}
In this section we will describe how we have achieved a model for our problem.

\subsection{Conceptual Model}
\noindent We started by discussing the semantics of main elements: \emph{what
does a bicycle mean?} \emph{what does it consist of?} and \emph{how are they
tied together?} We have answered the questions using sentences like ``A
\emph{BICYCLE} is made up of bicycle \emph{PARTS}. Parts can be customized with
a \emph{COLOR} and they can be provided by different \emph{MANUFACTURERS}.
A bicycle part can be a \emph{FRAME}, \emph{WHEEL}, etc''. After describing the
bicycle in sentences, we have created a graphical representation to give us a
more easy-to-understand overview. The model is depicted in figure
\ref{fig.conceptual_model} and it represents the \emph{conceptual model}.

\subsection{Logical Model}
\noindent After defining the \emph{conceptual model}, our aim was to make it
more concrete. This was achieved by defining a \emph{logical model} and
implementing it in EMF. Models in EMF are represented as \emph{Ecore} which is 
an implementation of \emph{eMOF} (Essential MOF\footnote{Meta-Object Facility}) meta-meta model. The resulting \emph{Ecore} model represents the \emph{meta-model} for
the NBS' custom bicycles. The meta-model will be used to model customized bicycles, which can be constructed using the
available bicycle parts. Relating to our \emph{logical model}, we have
implemented the \emph{Ecore} model as described in figure \ref{fig.ecore_model}.
We have defined the \emph{Node} as our top level abstraction for any element
that can describe a bicycle, which is further specialized by the \emph{Part}
abstract class. Having the \emph{Node} as top-level element, make the design
more flexible, then having only the \emph{Part} class. For instance,  in the
future, we could be asked to extend the model to support \emph{tuning}. Although
the items used for tuning will be ``part'' of the bicycle, they do not have the
same semantics as the physical \emph{Part}s that a bicycle is made up of.
Therefore, we can extend the \emph{Node} to create a \emph{TuningItem} class to
serve as an abstraction for bicycle tuning item.\\

\noindent The \emph{Part} class defines the attributes that all specific bicycle
parts have in common: \emph{price}, \emph{manufacturer} and \emph{id}. The
manufacturer is represented by the \emph{Manufacturers} enumeration, which
defines three possible manufacturers the NBS currently is working with:
\emph{SuperParts}, \emph{HandMadeParts} and \emph{HomeParts}. New manufacturers
can easily be added to the enumeration.\\

\noindent Currently, the following concrete bicycle parts are available to
create a custom bike: \emph{Frame}, \emph{Gear}, \emph{Wheel}, \emph{Pedals}, \emph{Saddle},
\emph{HandBreaks} and \emph{HandleBars}. The frame is characterize by
a color and a \emph{FrameType}, which we represented as an enumeration defining
values for the currently supported frame types \emph{RacerFrame}, \emph{CityFrame} and
\emph{UniciclyFrame}, each for a specific category of bicycle: RacerBike,
CityBike and Unicycle. The \emph{Gear} class represents an abstract class for
the two types of gears a bicycle can have: \emph{InnerGear} and
\emph{OuterGear}. The pedals and saddle can also be of different types,
depending on the customers needs. The available types are defined as
enumerations, such as \emph{PedalType} and \emph{SaddleType}. \\

\noindent In the conceptual model we have thought that each part should be
described by a color. In the current implementation we have added this attribute only to the frame, but if
needed, the model can easily be modified to support color definition for other parts as well.
In the design phase we aimed for the possibility to specify any custom color for
the frame, but in the implementation phase enumerating the different colors was
found to be the most suitable solution. In order to provide more flexibility to
the users of the system a \emph{Color} could have been represented by the three
RGB (red-green-blue) values in a 0-255 range, but it would have required an
additional tool to provide RGB value to a certain color, this feature being out
of the scope of the current work.\\

\noindent The custom bicycle is represented as the \emph{Bicycle} class, which
is made up of many \emph{Part}s and has information about the \emph{total
price}, \emph{delivery date} and the \emph{customer data}. The bicycle parts
that define a custom bicycle are stored in the \emph{parts} attribute of
the \emph{Bicycle} class, which is modeled as a composition association from the
\emph{Bicycle} to the \emph{Part} class, with the multiplicity 0..*. The customer data is represented as the \emph{CustomerData}
class with the name and address parameters. In order to support more data than
the essential ones defined as attributes, we have created the \emph{Parameter}
class characterized by a name and a value. Both the Bicycle and the CustomerData
are linked through a containment association with the Parameters class.\\


\subsection{Custom Bicycle Models}
Earlier in this section, we have discussed that EMF contains the \emph{Ecore}
meta-model. Actually, there is another meta-model EMF is based on:
the \emph{Genmodel}. The \emph{Ecore} meta-model contains the information about
the defined classes, while the genmodel contains additional information for
code generation. In order to allow the creation of custom bikes based on our
meta-model, we had to create the genmodel based on the existing model defined as
\emph{Ecore}. Using the genmodel we generated the \emph{model code} and the
\emph{edit}, \emph{editor} and \emph{test} plug-ins. We started a new runtime
environment with the generated plug-ins, and we have modeled three custom bicycles
\emph{CityBike}, \emph{SportsBike} and \emph{Unicycle}. The modeled bicycles are
illustrated in appendix \ref{appendix.emf_models}. The bikes are modeled as
\emph{.nowarebicycleshop} model files. While modeling a bicycle, the bike
\emph{editor} presents all the elements that can be added to the model. The
properties of different elements can be set in the properties view.\\

\noindent In order to compute the total price of a custom bicycle and to define
prices for parts, we had to modify the generated model code. In the
\emph{PartsImpl} abstract class we have attached the value
10 to the \emph{PRICE\_EDEFAULT} attribute and annotated it with
\emph{@generated NOT} so the generator does not override our modification. This
means that each bicycle part will have 10 as default price. In the
\emph{BicycleImpl} class we have modified \emph{getPrice()} method to return the
sum of all the contained parts' prices. This method was also annotated with
\emph{@generated NOT}.

\subsection{Modeling Layers}
\noindent As already discussed at the beginning of this section, the Eclipse
Modeling Framework (EMF) is based on \emph{eMOF}. To create our \emph{Ecore} model,
we had to instantiate elements from the \emph{Ecore} meta-meta model:
\emph{EClass}, \emph{EReference}, \emph{EAttribute} and \emph{EDataType}. To create custom
bicycles, elements in our meta-model have to be instantiated in a more concrete
model. An explanation on the modeling levels and how models conform to each
other is depicted in figure \ref{fig.metamodel_layers_schema}:
\begin{itemize}
  \item{M3} is the EMF meta meta-model which is based on eMOF used to
  define other meta-models; it is called meta meta-model because it can be used to
  represent itself, being its own meta-model
  \item{M2} is the meta-model for modeling customized bikes, represented by
  our \emph{Ecore} model
  \item{M1} represents the custom bike model, so they are instances of the
  meta-model; this modeling layer represents the scenarios in which a customer
  or a sales clerk can define a custom bike (e.g. the custom model we defined
  in appendix \ref{appendix.emf_models})
  \item{M0} represents the data and concrete ordering of items; for example,
  customer ``Test Person'' orders a city bike with a red colored CityFrame
  provided by SuperParts, which costs 150 and wants it to be delivered by the
  8th of December, etc.
\end{itemize} 
