\section{Introduction}
\label{sec.introduction}
\noindent The assignment is about creating an IT system for a fictive bicycle
shop called \emph{Noware Bicycle Shop} (NBS), which specializes in selling
customized handmade bicycles and have been in business for several years as an
old-fashioned, physical shop. The main problem they are having is that it is
difficult to demonstrate the possibilities of customizing bicycles for the
customers just through magazines and catalogs. Secondly, the sales clerks spend
a great amount of their time on supporting customers to choose valid
configurations for their bikes. To alleviate the work of sales clerks and make
it easier for the customers to customize bikes, NBS would like to introduce an
IT solution with touch screen terminals where customers can easily create their
model of new bikes. The clerks will make use of a textual interface to rapidly
create bike configurations.

\subsection{Motivation}
\noindent The group's motivation for creating this task is to use the modeling
concepts learned through the Model Driven Development (MDD) lectures and apply
it on a realistic scenario. By completing the assignments the members will
have a better idea of how MDD can be used to model a physical shop and apply its
terms in an IT environment.  The process will also give a general idea of how
MDD can speed up the development process of the system.

\subsection{The Process}
\noindent The project is important because it guides us to apply the
technologies presented during the lectures, in order to gain a deeper
understanding of them. In order to fulfill our purpose, we are faced with a real
world scenario, which we described earlier in this section.\\

\noindent As the aim is to achieve a flexible and effective solution, we have
adopted a model driven development approach, which helps in capturing domain
knowledge described in terms of a \emph{Domain Specific Language} (DSL). This
kind of approach results into a closer cooperation between the users of the
system and the developers. More than that, the code is automatically generated
based on the model. In the remainder of this section we will briefly describe
our approach, relating to the main tasks we carried out:
\begin{itemize}
  \item construction of a conceptual data model to represent a single customized
  bicycle, to be delivered to a single customer
  \item construction and implementation of a logical model as an
  \emph{EMF}\footnote{Eclipse Modeling Framework} model
  \item definition of \emph{OCL}\footnote{Object Constraint Language}
  constraints
  \item development of a \emph{GMF}\footnote{Graphical Modeling Framework} based
  \emph{graphical DSL}
  \item development of a \emph{XText}\footnote{a Language Development Framework}
  based \emph{textual editor}
  \item creation of a \emph{M2T}\footnote{model to text} transformation using
  Xpand\footnote{a language specialized on code generation based on EMF models}
  to create a textural invoice that lists the bike configuration
\end{itemize} 

\noindent The first three tasks were mandatory. They represent the foundation
for the upcoming tasks. The EMF model represents the meta-model used for
defining the abstract syntax of the DSL. In order to assure a good structural
and semantical use of the DSL, constraints must be added to the model. We have
achieved this by using OCL. As the end user is both the customer and the sales
clerk, we have decided to implement both a textual and a graphical concrete syntax. On
one hand, the customer can use an intuitive and easy-to-use graphical
syntax, running on touch screen enabled devices that offer visualizations
of the model's instances to rapidly construct the desired bicycle. On the other
hand, the sales clerk can use the textual syntax to define instance models in a
text-based language, witch allows using extended features such as overriding
the price of bicycle parts. In order to represent the textual syntax, we have
created an \emph{EBNF}\footnote{Extended Backus Normal Form} grammar, which we later 
implemented in the XText editor. To implement the graphical syntax, we have used GMF, 
which maps the abstract syntax to a graphical notation for our DSL, resulting in
a feature rich diagram editor.\\

\noindent The report starts by describing the problem, the
motivation and the steps we have taken. Section \ref{sec.logical_model}
presents the way we have defined and implemented our models. Section
\ref{sec.ocl_constraints} presents the implementation of OCL constraints in the
model. Section \ref{sec.gmf} describes how we achieved the graphical syntax.
Section \ref{sec.xtext} presents the usage of XText in order to achieve textual
syntax. Section \ref{sec.m2t} describes the model to text transformation.
Finally, section \ref{sec.conclusions} concludes our work.
