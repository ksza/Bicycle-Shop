\section{Model to Text Transformation using XPand}
\label{sec.m2t}
\noindent In the implementation of NBS, we enable the generation of invoice for
a custom bicycle. The invoice will contain the customer details, delivery date,
the total price, a detailed list of the bicycle parts and additional parameters.
In order to achieve this functionality we have implemented a transformation from
a bicycle model to a textual representation in the HTML format. The feature is
accessible from the graphical editor and can be invoked by selecting the option
``Edit/Duplicate'' from the bicycle diagram contextual menu. This will generate
the current configuration invoice and open it in a browser. The above described
scenario is depicted in figure \ref{fig.xpand.usecase}. The code for the
extensions template is shown in appendix
\ref{appendix.xpand.implementation}.\\


\noindent To implement the M2T transformation, we have used \emph{XPand}. We
have implemented the templates in the same project with the EMF model. To be able to
do this, the first step was to enable the Xtend/Xpand nature on the model
project. Afterwards, we defined the two template files, \emph{Extensions.ext}
and \emph{InvoiceTemplate.xpt}, in the \emph{model/template} package. This way
both templates are able to import and work with the \emph{nowarebicycleshop}
model. Figure \ref{fig.xpand.model_xpand_invoice} illustrates the Xpand template
for our \emph{Eore} meta-model and an invoice as a result of M2T transformation
applied to the presented concrete model.\\

\noindent \emph{InvoiceTemplate.xpt} is a Xpand template where we define the
transformation from a custom bicycle model to an invoice. In the template file
we define three templates: \emph{main}, \emph{part} and \emph{param}. The main
template creates the file with the name provided by the \emph{getTitle()}
utility method and fills it with the following:
\begin{itemize}
  \item header for HTML document, provided by \emph{getDocumentHeader()}
  utility method
  \item customer name, address, delivery date and total price
  \item calls the \emph{part} template to fill the information for each bicycle
  part
  \item calls the \emph{param} template to fill the information for each
  additional parameter
  \item footer for HTML document, provided by \emph{getDocumentFooter()}
  method\\
\end{itemize}

\noindent The utility methods are defined in a Xtend template
\emph{Extensions.ext} (Figure \ref{lst.extensions}).\\

\noindent After implementing the templates, we have created the
\emph{dk.itu.mdd.nbs.diagram.custom} plug-in as a GMF extension point, in order
to get notified when an edit action occurs on the \emph{bicycle diagram}. The
functionality is implemented in the \emph{EditPolicyProvider} and
\emph{OpenEditPolicy} classes. The code and relevant comments are shown in
appendix \ref{appendix.xpand.implementation}. When the \emph{DuplicateRequest} request command is received,
the following actions are taken to fulfill the transformation:
\begin{itemize}
  \item get the Bicycle instance under the edited diagram
  \item configure the outlet to be used by the generator
  \item create an execution context for the XPand and XTend programmatic
  execution
  \item create the execution context based on our custom Outlet and
  mocked variables map
  \item create a mock EMFRegistry to contain our meta-model and the
  ecore meta-model
  \item register the mock registry in the execution context
  \item launch the M2T transformation using the \emph{XpandFacade} class
  \item show the invoice in an external browser\\
\end{itemize}

\noindent We have gained deeper technical background about Xtend/Xpand from
\url{http://www.peterfriese.de/using-xpand-in-your-eclipse-wizards/}.
