\section{Model to Text Transformation using XPand}
\label{sec.m2t}
In the implementation of our NBS, we offer the customer the possibility of
generation an invoice for his custom bicycle. The invoice will contain the name
and address of the customer, desired delivery date, total price of the bicycle
and a detailed list of the parts the bike is composed of. In order to do achieve
this functionality we have decided to implement a transformation from a custom
bicycle model to a textual representation (we have decided on the \emph{HTML}
format). The feature is offered only for the customers of the NBS, therefore
accessible from the graphical editor. When the user has finished
creating the custom bike and would like to finalize the order, he can invoke the
contextual menu on the bicycle diagram and select the option ``Edit/Duplicate''.
This will generate an invoice in HTML based on the current configuration and
will automatically open it in a browser. The above described scenario is depicted
in figure \ref{fig.xpand.usecase}. The code for the extensions template is
shown in appendix \ref{appendix.xpand.implementation}.

To implement the M2T Transformation, we have used \emph{XPand}. We have
implemented the templates in the same project with the EMF model. To be able to
do this, the first step was to enable the Xtent/Xpand nature on the model
project. Afterwards, we defined the two template files, \emph{Extensions.ext}
and \emph{InvoiceTemplate.xpt}, in the \emph{model/template} package. This way
both templates are able to import and work with the \emph{nowarebicycleshop}
model. Figure \ref{fig.xpand.model_xpand_invoice} illustrates the
InvoiceTemplate file and the resulting invoice a M2T transformation applied to
the presented concrete model. A snippet from the ecore model diagram is also
shown in the figure to create a clearer overview of how the elements are
related.\\

\emph{Extensions.ext} is an Xtent template which defines three utility
methods for the Xpand template: \emph{getTitle()} which creates the file name
for the invoice, \emph{getDocumentHeader()} to crate the HTML header and
\emph{getDocumentFooter} to create the HTML footer. \emph{InvoiceTemplate.xpt}
is a Xpand template where we define the actual transformation from the custom
bicycle model to an invoice represented as HTML. In the template file we define
two templates: \emph{main} and \emph{part}. The main template creates the file
with the name provided by the \emph{getTitle()} method in the extension and
fills it with the followings:
\begin{itemize}
  \item document header provided by \emph{getDocumentHeader()}
  \item customer name, address, delivery date and total price
  \item for each part in the bicycle model, calls the \emph{part} template
  filling in information regarding each bicycle part
  \item document footer provided by \emph{getDocumentFooter}
\end{itemize}

After implementing the templates, we have created the
\emph{dk.itu.mdd.nbs.diagram.custom} plug-in as a GMF extension point, in order
to get notified when an edit action occurs on the \emph{bicycle diagram}. The
functionality is implemented in the \emph{EditPolicyProvider} and
\emph{OpenEditPolicy} classes. The code and relevant comments are shown in
appendix \ref{appendix.xpand.implementation}. When the \emph{DuplicateRequest} request command is received,
the following actions are taken to fulfill the transformation:
\begin{itemize}
  \item get the Bicycle instance under the edited diagram
  \item configure the outlet to be used by the generator
  \item create an execution context for the XPand and XTend programmatic
  execution
  \item create the execution context based on our custom Outlet and
  mocked variables map
  \item create a mock EMFRegistry to contain our meta-model and the
  ecore meta-model
  \item register the mock registry in the execution context
  \item launch the M2T transformation using the \emph{XpandFacade} class
  \item show the invoice in an external browser
\end{itemize}

We have gained deeper technical background about Xtent/Xpand from
\cite{XPAND_EXAMPLE}.
