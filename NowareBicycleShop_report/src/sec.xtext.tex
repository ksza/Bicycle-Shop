\section{Textual DSL using XText}
\label{sec.xtext}
\noindent To help the sales clerk in their everyday tasks we have developed a
textual DSL using \emph{XText} framework. A textual language can speed up the process of
creating bicycle models and user data. Because of the nature of text files,
sales clerk can easily copy and edit already created model instances.\\

\noindent We have used an Extended Backus-Naur Form (EBNF)
meta-language notations to describe a textual DSL. Our textual DSL is external,
meaning that it is new, not extending an existing language. As we have
identified the entities, attributes and relations in the \emph{Ecore} meta-model, we use
\emph{XText} to generate the textual DSL. It creates a grammar with rules for
our entities, abstract types, properties, relations, etc. We build upon this
grammar and modify it for our needs. The final \emph{XText} grammar definition
is listed in figure \ref{appendix.xtext}. It is extensible and context-free.\\

\noindent The target user for the language is the sales clerk and he might not
have any programming experience, so we want the language to seem
natural. Therefore statement terminators, such as symbols or line feeds, are not
used. The grammar is rich, having types for entities, attribute
names, enumerators and other details. This is used to provide context assistance
in the tools.\\

\noindent When using the textual DSL, the editor executes syntax checking
(figure \ref{fig.dsl_empty_model}) and suggests code completion (figure
\ref{fig.dsl_autocomplete_parts} and \ref{fig.dsl_autocomplete_values}). The
user also gets help on what parts are left to define. The grammar is simple and
fast to enter while keeping it perspicuous. Excluding separators from the
grammar further simplifies the layout. Moreover, the keywords and class names
are differentiated with syntax highlighting. We have depicted three custom
bicycles modeled in the the textual editor in figure \ref{fig.xtext_city_bike},
\ref{fig.xtext_sport_model} and \ref{fig.xtext_unicycle_model}.