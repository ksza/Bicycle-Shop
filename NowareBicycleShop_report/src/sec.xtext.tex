\section{Textual DSL using XText}
\label{sec.xtext}

\noindent
To help the sales clerk in their everyday tasks we have developed a textual DSL
using Xtext technology. A textual language can speed up the process of creating
bicycle models, user and other data. Because of the nature of text files,
sales clerk can easily copy and edit already created model instances.\\

\noindent
To describe a DSL we have used an Extended Backus�Naur Form (EBNF)
metalanguage notations. Our DSL is external, meanig that it is completely
new, not extending an existing language. As we have identified the entities,
attributes and relations in our ecore model, we use Xtext to generate the DSL. It creates a grammar with rules
for our entities, abstract types, propteries, relations, etc. We build upon this grammar and modify it for our
needs. The final Xtext grammar definition is listed in Figure
\ref{appendix.xtext}. It is extendible and context-free.\\

\noindent
The target user for the language is the sales clerk and he might not
necessarily have any programming experience, so we want the language to seem
natural and helpful for the user. Therefore statement terminators, such as symbols or line feeds, are not
used. The grammar is rich, having types for entities, attribute
names, enumerators and other details. This is used to provide context assistance
in the tools.\\

\noindent
When using our DSL, the editor executes syntax checking (Figure
\ref{fig.dsl_empty_model}) and suggests code completion (Figures \ref{fig.dsl_autocomplete_parts},
\ref{fig.dsl_autocomplete_values}), so the user does not have to type full words and can get help on what parts are left to define.
The grammar is simple, so it is fast to enter while keeping it perspicuous.
Excluding separators from the grammar further simplifies the layout.
Moreover, the keywords and class names are differentiated with syntax
highlighting.\\
